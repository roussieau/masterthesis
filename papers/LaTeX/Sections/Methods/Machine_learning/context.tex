\section{Context}
In the previous sections, we defined how we managed to select specific data for malware classification, extract them and then treat them in order to get consistent materials. This process has now led us towards another goal which is machine learning, i.e. the interpretation of our data to build a model which will eventually make prediction with a certain accuracy.

The thing is that out there, in the beautiful world of artificial intelligence and knowledge engineering, a rookie could easily get lost in the diversity of possibilities offered. Fortunately, plenty of APIs and other libraries exist to facilitate the hard work of understanding and selecting the best combination of learnings.

In the following sections, we will start with the description of the type of learning we decided to select for the creation of our models. Then, our selection of classifiers will be detailed to finally end up with the set of experiments that were run with these classifiers on our datasets. The goal of this last step is to identify which classifier, after being properly tuned, is the best suited to determine if a sample is possibly packed.